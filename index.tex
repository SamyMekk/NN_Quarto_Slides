% Options for packages loaded elsewhere
\PassOptionsToPackage{unicode}{hyperref}
\PassOptionsToPackage{hyphens}{url}
%
\documentclass[
  ignorenonframetext,
]{beamer}
\usepackage{pgfpages}
\setbeamertemplate{caption}[numbered]
\setbeamertemplate{caption label separator}{: }
\setbeamercolor{caption name}{fg=normal text.fg}
\beamertemplatenavigationsymbolsempty
% Prevent slide breaks in the middle of a paragraph
\widowpenalties 1 10000
\raggedbottom
\setbeamertemplate{part page}{
  \centering
  \begin{beamercolorbox}[sep=16pt,center]{part title}
    \usebeamerfont{part title}\insertpart\par
  \end{beamercolorbox}
}
\setbeamertemplate{section page}{
  \centering
  \begin{beamercolorbox}[sep=12pt,center]{section title}
    \usebeamerfont{section title}\insertsection\par
  \end{beamercolorbox}
}
\setbeamertemplate{subsection page}{
  \centering
  \begin{beamercolorbox}[sep=8pt,center]{subsection title}
    \usebeamerfont{subsection title}\insertsubsection\par
  \end{beamercolorbox}
}
\AtBeginPart{
  \frame{\partpage}
}
\AtBeginSection{
  \ifbibliography
  \else
    \frame{\sectionpage}
  \fi
}
\AtBeginSubsection{
  \frame{\subsectionpage}
}

\usepackage{amsmath,amssymb}
\usepackage{iftex}
\ifPDFTeX
  \usepackage[T1]{fontenc}
  \usepackage[utf8]{inputenc}
  \usepackage{textcomp} % provide euro and other symbols
\else % if luatex or xetex
  \usepackage{unicode-math}
  \defaultfontfeatures{Scale=MatchLowercase}
  \defaultfontfeatures[\rmfamily]{Ligatures=TeX,Scale=1}
\fi
\usepackage{lmodern}
\ifPDFTeX\else  
    % xetex/luatex font selection
\fi
% Use upquote if available, for straight quotes in verbatim environments
\IfFileExists{upquote.sty}{\usepackage{upquote}}{}
\IfFileExists{microtype.sty}{% use microtype if available
  \usepackage[]{microtype}
  \UseMicrotypeSet[protrusion]{basicmath} % disable protrusion for tt fonts
}{}
\makeatletter
\@ifundefined{KOMAClassName}{% if non-KOMA class
  \IfFileExists{parskip.sty}{%
    \usepackage{parskip}
  }{% else
    \setlength{\parindent}{0pt}
    \setlength{\parskip}{6pt plus 2pt minus 1pt}}
}{% if KOMA class
  \KOMAoptions{parskip=half}}
\makeatother
\usepackage{xcolor}
\newif\ifbibliography
\setlength{\emergencystretch}{3em} % prevent overfull lines
\setcounter{secnumdepth}{-\maxdimen} % remove section numbering


\providecommand{\tightlist}{%
  \setlength{\itemsep}{0pt}\setlength{\parskip}{0pt}}\usepackage{longtable,booktabs,array}
\usepackage{calc} % for calculating minipage widths
\usepackage{caption}
% Make caption package work with longtable
\makeatletter
\def\fnum@table{\tablename~\thetable}
\makeatother
\usepackage{graphicx}
\makeatletter
\newsavebox\pandoc@box
\newcommand*\pandocbounded[1]{% scales image to fit in text height/width
  \sbox\pandoc@box{#1}%
  \Gscale@div\@tempa{\textheight}{\dimexpr\ht\pandoc@box+\dp\pandoc@box\relax}%
  \Gscale@div\@tempb{\linewidth}{\wd\pandoc@box}%
  \ifdim\@tempb\p@<\@tempa\p@\let\@tempa\@tempb\fi% select the smaller of both
  \ifdim\@tempa\p@<\p@\scalebox{\@tempa}{\usebox\pandoc@box}%
  \else\usebox{\pandoc@box}%
  \fi%
}
% Set default figure placement to htbp
\def\fps@figure{htbp}
\makeatother

\makeatletter
\@ifpackageloaded{tcolorbox}{}{\usepackage[skins,breakable]{tcolorbox}}
\@ifpackageloaded{fontawesome5}{}{\usepackage{fontawesome5}}
\definecolor{quarto-callout-color}{HTML}{909090}
\definecolor{quarto-callout-note-color}{HTML}{0758E5}
\definecolor{quarto-callout-important-color}{HTML}{CC1914}
\definecolor{quarto-callout-warning-color}{HTML}{EB9113}
\definecolor{quarto-callout-tip-color}{HTML}{00A047}
\definecolor{quarto-callout-caution-color}{HTML}{FC5300}
\definecolor{quarto-callout-color-frame}{HTML}{acacac}
\definecolor{quarto-callout-note-color-frame}{HTML}{4582ec}
\definecolor{quarto-callout-important-color-frame}{HTML}{d9534f}
\definecolor{quarto-callout-warning-color-frame}{HTML}{f0ad4e}
\definecolor{quarto-callout-tip-color-frame}{HTML}{02b875}
\definecolor{quarto-callout-caution-color-frame}{HTML}{fd7e14}
\makeatother
\makeatletter
\@ifpackageloaded{caption}{}{\usepackage{caption}}
\AtBeginDocument{%
\ifdefined\contentsname
  \renewcommand*\contentsname{Table of contents}
\else
  \newcommand\contentsname{Table of contents}
\fi
\ifdefined\listfigurename
  \renewcommand*\listfigurename{List of Figures}
\else
  \newcommand\listfigurename{List of Figures}
\fi
\ifdefined\listtablename
  \renewcommand*\listtablename{List of Tables}
\else
  \newcommand\listtablename{List of Tables}
\fi
\ifdefined\figurename
  \renewcommand*\figurename{Figure}
\else
  \newcommand\figurename{Figure}
\fi
\ifdefined\tablename
  \renewcommand*\tablename{Table}
\else
  \newcommand\tablename{Table}
\fi
}
\@ifpackageloaded{float}{}{\usepackage{float}}
\floatstyle{ruled}
\@ifundefined{c@chapter}{\newfloat{codelisting}{h}{lop}}{\newfloat{codelisting}{h}{lop}[chapter]}
\floatname{codelisting}{Listing}
\newcommand*\listoflistings{\listof{codelisting}{List of Listings}}
\makeatother
\makeatletter
\makeatother
\makeatletter
\@ifpackageloaded{caption}{}{\usepackage{caption}}
\@ifpackageloaded{subcaption}{}{\usepackage{subcaption}}
\makeatother

\usepackage{bookmark}

\IfFileExists{xurl.sty}{\usepackage{xurl}}{} % add URL line breaks if available
\urlstyle{same} % disable monospaced font for URLs
\hypersetup{
  pdftitle={Slides Test},
  pdfauthor={Samy Mekkaoui},
  hidelinks,
  pdfcreator={LaTeX via pandoc}}


\title{Slides Test}
\subtitle{CMAP Ecole Polytechnique}
\author{Samy Mekkaoui}
\date{}

\begin{document}
\frame{\titlepage}


\section{In the morning}\label{in-the-morning}

\begin{frame}{Contexte}
\phantomsection\label{contexte}
\begin{itemize}[<+->]
\tightlist
\item
  {\textbf{Qui sommes-nous ?}}

  \begin{itemize}[<+->]
  \tightlist
  \item
    des {\textbf{data scientists}} de l'Insee
  \item
    frustrés par l'{\textbf{approche}} souvent purement
    {\textbf{technique}} de la data science
  \item
    convaincus que les {\textbf{bonnes pratiques de développement}}
    valent à être enseignées
  \end{itemize}
\end{itemize}
\end{frame}

\begin{frame}{Qu'est ce qu'un data scientist ?}
\phantomsection\label{quest-ce-quun-data-scientist}
\begin{center}
\includegraphics[width=\linewidth,height=2.60417in,keepaspectratio]{img/sexiest-job.png}
\end{center}

\begin{itemize}[<+->]
\item
  Tendance à la {\textbf{spécialisation}} : \emph{data analyst},
  \emph{data engineer}, \emph{ML Engineer}\ldots{}
\item
  Rôle d'{\textbf{interface}} entre métier et équipes techniques

  \begin{itemize}[<+->]
  \tightlist
  \item
    {\textbf{Compétences mixtes}} : savoir métier, modélisation, IT
  \end{itemize}
\end{itemize}
\end{frame}

\begin{frame}{Getting up}
\phantomsection\label{getting-up}
\begin{itemize}[<+->]
\tightlist
\item
  Let's put some maths here to check if it works :
\end{itemize}

\[
f(x) = \int_{0}^{T} g(s,x) ds
\]
\end{frame}

\begin{frame}{La notion de mise en production}
\phantomsection\label{la-notion-de-mise-en-production}
\begin{itemize}[<+->]
\tightlist
\item
  {\textbf{Mettre en production}} : faire {\textbf{vivre}} une
  application dans l'espace de ses {\textbf{utilisateurs}}

  \begin{itemize}[<+->]
  \tightlist
  \item
    Notion simple mais mise en oeuvre compliquée !
  \end{itemize}
\item
  {\textbf{Dépasser le stade de l'expérimentation}}

  \begin{itemize}[<+->]
  \tightlist
  \item
    Comprendre les {\textbf{besoins}} des utilisateurs
  \item
    {\textbf{Bonnes pratiques}} de développement
  \item
    Techniques informatiques d'{\textbf{industrialisation}}
  \end{itemize}
\end{itemize}
\end{frame}

\begin{frame}[fragile]{Contenu du cours}
\phantomsection\label{contenu-du-cours}
\begin{itemize}[<+->]
\tightlist
\item
  {\textbf{Pré-requis}}

  \begin{itemize}[<+->]
  \tightlist
  \item
    Introduction au terminal \texttt{Linux}
  \item
    {\textbf{Contrôle de version}} avec \texttt{Git}
  \end{itemize}
\item
  {\textbf{Bonnes pratiques}} de développement

  \begin{itemize}[<+->]
  \tightlist
  \item
    {\textbf{Travail collaboratif}} avec \texttt{Git}
  \item
    {\textbf{Qualité}} du code
  \item
    {\textbf{Structure}} des projets
  \item
    Traitement des {\textbf{données volumineuses}}
  \item
    Favoriser la {\textbf{portabilité}} d'une application
  \end{itemize}
\item
  {\textbf{Mise en production}}

  \begin{itemize}[<+->]
  \tightlist
  \item
    {\textbf{Déploiement}}
  \item
    {\textbf{MLOps}}
  \end{itemize}
\end{itemize}
\end{frame}

\section{\texorpdfstring{Partie {1️⃣} : bonnes pratiques de
développement}{Partie 1️⃣ : bonnes pratiques de développement}}\label{partie-1-bonnes-pratiques-de-duxe9veloppement}

\begin{frame}[fragile]{Plan de la partie}
\phantomsection\label{plan-de-la-partie}
{1️⃣} {\textbf{Travail collaboratif}} avec \texttt{Git}

{2️⃣} {\textbf{Qualité}} du code

{3️⃣} {\textbf{Structure}} des projets

{4️⃣} Traitement des {\textbf{données volumineuses}}

{5️⃣} Favoriser la {\textbf{portabilité}} d'une application
\end{frame}

\section{\texorpdfstring{{1️⃣} Le travail collaboratif avec
\texttt{Git}}{1️⃣ Le travail collaboratif avec Git}}\label{le-travail-collaboratif-avec-git}

\begin{frame}{Pourquoi utiliser Git ?}
\phantomsection\label{pourquoi-utiliser-git}
\begin{figure}[H]

{\centering \includegraphics[width=\linewidth,height=4.94792in,keepaspectratio]{img/timeline.png}

}

\caption{Source :
\href{https://thinkr.fr/travailler-avec-git-via-rstudio-et-versionner-son-code/}{ThinkR}}

\end{figure}%
\end{frame}

\begin{frame}{Concepts essentiels}
\phantomsection\label{concepts-essentiels}
\begin{figure}[H]

{\centering \includegraphics[width=\linewidth,height=4.16667in,keepaspectratio]{img/gitallinone.png}

}

\caption{Source :
\href{http://fabacademy.org/2021/labs/bhubaneswar/students/deepak-chaudhry/ia_PPFP.html}{fabacademy.org}}

\end{figure}%
\end{frame}

\begin{frame}[fragile]{Bonnes pratiques}
\phantomsection\label{bonnes-pratiques}
\textbf{Que versionne-t-on ?}

\begin{itemize}[<+->]
\tightlist
\item
  Essentiellement du {\textbf{code source}}
\item
  {\textbf{Pas d'outputs}} (fichiers \texttt{.html}, \texttt{.pdf},
  modèles\ldots)
\item
  {\textbf{Pas de données}}, d'informations locales ou sensibles
\end{itemize}

\pause

\begin{tcolorbox}[enhanced jigsaw, toprule=.15mm, leftrule=.75mm, title=\textcolor{quarto-callout-note-color}{\faInfo}\hspace{0.5em}{Note}, titlerule=0mm, breakable, colframe=quarto-callout-note-color-frame, colback=white, bottomrule=.15mm, rightrule=.15mm, toptitle=1mm, coltitle=black, opacitybacktitle=0.6, opacityback=0, colbacktitle=quarto-callout-note-color!10!white, bottomtitle=1mm, left=2mm, arc=.35mm]

Pour définir des règles qui évitent de committer tel ou tel fichier, on
utilise un fichier nommé \textbf{\texttt{.gitignore}}.

Si on mélange du code et des éléments annexes (\emph{output},
données\ldots) dans un même dossier, il {\textbf{faut consacrer du temps
à ce fichier}}.

Le site
\href{https://www.toptal.com/developers/gitignore}{\texttt{gitignore.io}}
peut vous fournir des modèles.

N'hésitez pas à y ajouter des règles conservatrices (par exemple
\texttt{*.csv}), comme cela est expliqué dans
\href{https://www.book.utilitr.org/git.html?q=gitignore\#gitignore}{la
documentation \texttt{utilitR}}.

\end{tcolorbox}
\end{frame}

\begin{frame}{Bonnes pratiques}
\phantomsection\label{bonnes-pratiques-1}
\end{frame}

\begin{frame}{Breakfast}
\phantomsection\label{breakfast}
\begin{itemize}[<+->]
\tightlist
\item
  Eat eggs
\item
  Drink coffee
\end{itemize}
\end{frame}

\section{In the evening}\label{in-the-evening}

\begin{frame}{Dinner}
\phantomsection\label{dinner}
\begin{itemize}[<+->]
\tightlist
\item
  Eat spaghetti
\item
  Drink wine
\end{itemize}
\end{frame}

\begin{frame}{Going to sleep}
\phantomsection\label{going-to-sleep}
\begin{itemize}[<+->]
\tightlist
\item
  Get in bed
\item
  Count sheep
\end{itemize}
\end{frame}




\end{document}
